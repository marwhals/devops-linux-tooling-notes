%! Author = Marjan
%! Date = 16/03/2025

% Preamble
\documentclass[11pt]{article}

\usepackage[a4paper,margin=1in]{geometry}
\usepackage{xcolor}
\usepackage{listings}
\usepackage{tgheros} % Modern sans-serif font
\renewcommand{\familydefault}{\sfdefault} % Use sans-serif as default

\definecolor{codegray}{gray}{0.9}
\lstdefinestyle{kubeStyle}{
    backgroundcolor=\color{codegray},
    basicstyle=\ttfamily\footnotesize,
    keywordstyle=\color{blue},
    commentstyle=\color{gray},
    breaklines=true,
    frame=single,
    showstringspaces=false
}

\title{\textbf{Kubernetes Cheat Sheet}}
\author{}
\date{\today}

\begin{document}
    \maketitle

    \section*{Cluster Management}
    \begin{lstlisting}[style=kubeStyle]
# View cluster info
kubectl cluster-info

# View all nodes
kubectl get nodes

# Describe a node
kubectl describe node <node_name>

# Get cluster events
kubectl get events

# View configuration
kubectl config view
    \end{lstlisting}

    \section*{Working with Pods}
    \begin{lstlisting}[style=kubeStyle]
# List all pods
kubectl get pods

# Get pods with more details
kubectl get pods -o wide

# Get detailed pod information
kubectl describe pod <pod_name>

# Delete a pod
kubectl delete pod <pod_name>

# Run a temporary pod
kubectl run mypod --image=nginx --restart=Never
    \end{lstlisting}

    \section*{Deployments and Services}
    \begin{lstlisting}[style=kubeStyle]
# List deployments
kubectl get deployments

# Create a deployment
kubectl create deployment my-deployment --image=nginx

# Scale a deployment
kubectl scale deployment <deployment_name> --replicas=3

# Update a deployment
kubectl set image deployment/<deployment_name> nginx=nginx:latest

# Rollback a deployment
kubectl rollout undo deployment/<deployment_name>

# Expose a deployment as a service
kubectl expose deployment <deployment_name> --type=NodePort --port=80
    \end{lstlisting}

    \section*{Working with Namespaces}
    \begin{lstlisting}[style=kubeStyle]
# List all namespaces
kubectl get namespaces

# Create a namespace
kubectl create namespace my-namespace

# Delete a namespace
kubectl delete namespace my-namespace
    \end{lstlisting}

    \section*{ConfigMaps and Secrets}
    \begin{lstlisting}[style=kubeStyle]
# Create a ConfigMap from a file
kubectl create configmap my-config --from-file=config.yaml

# View ConfigMaps
kubectl get configmaps

# Create a Secret from literals
kubectl create secret generic my-secret --from-literal=username=admin --from-literal=password=secret

# View Secrets
kubectl get secrets
    \end{lstlisting}

    \section*{Logs and Debugging}
    \begin{lstlisting}[style=kubeStyle]
# View logs of a pod
kubectl logs <pod_name>

# Stream logs from a pod
kubectl logs -f <pod_name>

# Get into a running pod
kubectl exec -it <pod_name> -- /bin/sh

# Debug a pod with a temporary container
kubectl debug <pod_name> --image=busybox --target=nginx
    \end{lstlisting}

    \section*{Persistent Volumes}
    \begin{lstlisting}[style=kubeStyle]
# List Persistent Volumes (PVs)
kubectl get pv

# List Persistent Volume Claims (PVCs)
kubectl get pvc

# Describe a PVC
kubectl describe pvc <pvc_name>
    \end{lstlisting}

    \section*{Networking}
    \begin{lstlisting}[style=kubeStyle]
# Get all services
kubectl get services

# Get service details
kubectl describe service <service_name>

# Get network policies
kubectl get networkpolicies
    \end{lstlisting}

    \section*{Helm (Kubernetes Package Manager)}
    \begin{lstlisting}[style=kubeStyle]
# Install a Helm chart
helm install my-release bitnami/nginx

# List installed Helm releases
helm list

# Upgrade a Helm release
helm upgrade my-release bitnami/nginx

# Uninstall a Helm release
helm uninstall my-release
    \end{lstlisting}

\end{document}
