%! Author = Marjan
%! Date = 16/03/2025

% Preamble
\documentclass[11pt]{article}

\usepackage[a4paper,margin=1in]{geometry}
\usepackage{xcolor}
\usepackage{listings}
\usepackage{tgheros} % Modern sans-serif font
\renewcommand{\familydefault}{\sfdefault} % Use sans-serif as default

\definecolor{codegray}{gray}{0.9}
\lstdefinestyle{jenkinsStyle}{
    backgroundcolor=\color{codegray},
    basicstyle=\ttfamily\footnotesize,
    keywordstyle=\color{blue},
    commentstyle=\color{gray},
    breaklines=true,
    frame=single,
    showstringspaces=false
}

\title{\textbf{Jenkins Cheat Sheet}}
\author{}
\date{\today}

\begin{document}
    \maketitle

    \section*{Jenkins CLI Commands}
    \begin{lstlisting}[style=jenkinsStyle]
# List installed plugins
java -jar jenkins-cli.jar -s http://localhost:8080 list-plugins

# Start a Jenkins job
java -jar jenkins-cli.jar -s http://localhost:8080 build my-job

# Get Jenkins system info
java -jar jenkins-cli.jar -s http://localhost:8080 system-info

# Restart Jenkins
java -jar jenkins-cli.jar -s http://localhost:8080 restart
    \end{lstlisting}

    \section*{Pipeline Syntax (Declarative)}
    \begin{lstlisting}[style=jenkinsStyle]
pipeline {
    agent any
    stages {
        stage('Build') {
            steps {
                sh 'echo Building...'
            }
        }
        stage('Test') {
            steps {
                sh 'echo Testing...'
            }
        }
        stage('Deploy') {
            steps {
                sh 'echo Deploying...'
            }
        }
    }
}
    \end{lstlisting}

    \section*{Pipeline Syntax (Scripted)}
    \begin{lstlisting}[style=jenkinsStyle]
node {
    stage('Build') {
        sh 'echo Building...'
    }
    stage('Test') {
        sh 'echo Testing...'
    }
    stage('Deploy') {
        sh 'echo Deploying...'
    }
}
    \end{lstlisting}

    \section*{Managing Jenkins Jobs}
    \begin{lstlisting}[style=jenkinsStyle]
# List all Jenkins jobs
java -jar jenkins-cli.jar -s http://localhost:8080 list-jobs

# Delete a job
java -jar jenkins-cli.jar -s http://localhost:8080 delete-job my-job

# Copy a job
java -jar jenkins-cli.jar -s http://localhost:8080 copy-job old-job new-job

# Create a job from XML
java -jar jenkins-cli.jar -s http://localhost:8080 create-job my-job < config.xml
    \end{lstlisting}

    \section*{Working with Plugins}
    \begin{lstlisting}[style=jenkinsStyle]
# Install a plugin
java -jar jenkins-cli.jar -s http://localhost:8080 install-plugin git

# Uninstall a plugin
java -jar jenkins-cli.jar -s http://localhost:8080 uninstall-plugin git

# List available plugins
java -jar jenkins-cli.jar -s http://localhost:8080 list-plugins | grep available
    \end{lstlisting}

    \section*{User Management}
    \begin{lstlisting}[style=jenkinsStyle]
# Create a user
java -jar jenkins-cli.jar -s http://localhost:8080 create-user --username admin --password secret

# List users
java -jar jenkins-cli.jar -s http://localhost:8080 list-users

# Delete a user
java -jar jenkins-cli.jar -s http://localhost:8080 delete-user admin
    \end{lstlisting}

    \section*{Jenkins System Configuration}
    \begin{lstlisting}[style=jenkinsStyle]
# Reload configuration
java -jar jenkins-cli.jar -s http://localhost:8080 reload-configuration

# Backup Jenkins configuration
tar -czvf jenkins-backup.tar.gz /var/lib/jenkins
    \end{lstlisting}

\end{document}
