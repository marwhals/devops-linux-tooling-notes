%! Author = Marjan
%! Date = 16/03/2025

% Preamble
\documentclass[11pt]{article}

\documentclass{article}
\usepackage[a4paper,margin=1in]{geometry}
\usepackage{xcolor}
\usepackage{listings}
\usepackage{tgheros} % Modern sans-serif font
\renewcommand{\familydefault}{\sfdefault} % Use sans-serif as default

\definecolor{codegray}{gray}{0.9}
\lstdefinestyle{linuxStyle}{
    backgroundcolor=\color{codegray},
    basicstyle=\ttfamily\footnotesize,
    keywordstyle=\color{blue},
    commentstyle=\color{gray},
    breaklines=true,
    frame=single,
    showstringspaces=false
}

\title{\textbf{Linux Cheat Sheet}}
\author{}
\date{\today}

\begin{document}
    \maketitle
    \section*{Grep - Searching Text}
    \begin{lstlisting}[style=linuxStyle]
# Search for a word in a file
grep "pattern" file.txt

# Case-insensitive search
grep -i "pattern" file.txt

# Search recursively in all files of a directory
grep -r "pattern" /path/to/dir

# Show line numbers in results
grep -n "pattern" file.txt

# Show only matching part
grep -o "pattern" file.txt

# Invert search (show lines NOT matching pattern)
grep -v "pattern" file.txt

# Use extended regex
grep -E "error|failed" log.txt

# Count occurrences of a pattern
grep -c "pattern" file.txt

# Highlight matches in color
grep --color=auto "pattern" file.txt

# Search for whole words only
grep -w "word" file.txt

# Grep from command output
ps aux | grep apache

# Find empty lines
grep "^$" file.txt

# Search multiple patterns
grep -e "error" -e "warning" log.txt
    \end{lstlisting}ó


    \section*{File & Directory Management}
    \begin{lstlisting}[style=linuxStyle]
# List files (detailed, hidden)
ls -lah

# Change directory
cd /path/to/directory

# Create a new directory
mkdir mydir

# Remove a file/directory
rm file.txt
rm -r mydir

# Copy files/directories
cp file1.txt file2.txt
cp -r dir1 dir2

# Move/rename files
mv oldname.txt newname.txt
mv file.txt /new/path/

# Show file contents
cat file.txt
less file.txt

# Create empty files
touch file.txt

# Find files
find /path -name "file.txt"
locate file.txt
    \end{lstlisting}

    \section*{Permissions & Ownership}
    \begin{lstlisting}[style=linuxStyle]
# Change file permissions (read, write, execute)
chmod 755 file.sh

# Change ownership
chown user:group file.txt

# Change file attributes
chattr +i file.txt   # Make file immutable
lsattr file.txt      # Show attributes
    \end{lstlisting}

    \section*{Process Management}
    \begin{lstlisting}[style=linuxStyle]
# List running processes
ps aux
top
htop  # Interactive process manager

# Kill a process
kill <PID>
kill -9 <PID>  # Force kill

# Background/Foreground jobs
command &    # Run in background
jobs         # List background jobs
fg %1        # Bring job 1 to foreground
bg %1        # Resume job 1 in background

# Monitor processes
watch -n 1 "ps aux | grep myprocess"
    \end{lstlisting}

    \section*{Networking Commands}
    \begin{lstlisting}[style=linuxStyle]
# Show IP addresses
ip a
ifconfig (deprecated)

# Show active connections
netstat -tulnp
ss -tulnp

# Ping a server
ping google.com

# Check open ports
nmap localhost

# Download files
wget http://example.com/file.zip
curl -O http://example.com/file.zip

# Test network speed
speedtest-cli

# Check DNS resolution
nslookup google.com
dig google.com
    \end{lstlisting}

    \section*{User Management}
    \begin{lstlisting}[style=linuxStyle]
# Add a new user
useradd -m -s /bin/bash newuser
passwd newuser

# Delete a user
userdel -r newuser

# List users
cat /etc/passwd

# Add a user to a group
usermod -aG sudo newuser

# Switch users
su - newuser
    \end{lstlisting}

    \section*{Disk & Storage Management}
    \begin{lstlisting}[style=linuxStyle]
# Show disk usage
df -h
du -sh /path/to/directory

# List mounted drives
lsblk
mount | column -t

# Mount/Unmount a drive
mount /dev/sdb1 /mnt/usb
umount /mnt/usb

# Check disk health
fsck /dev/sda
smartctl -a /dev/sda
    \end{lstlisting}

    \section*{System Monitoring}
    \begin{lstlisting}[style=linuxStyle]
# Show system uptime
uptime

# Show memory usage
free -h

# Show CPU info
lscpu

# Show system logs
journalctl -xe
tail -f /var/log/syslog
    \end{lstlisting}

    \section*{Package Management}
    \subsection*{Debian-based (Ubuntu, Debian)}
    \begin{lstlisting}[style=linuxStyle]
# Update package list
sudo apt update

# Upgrade packages
sudo apt upgrade

# Install a package
sudo apt install package-name

# Remove a package
sudo apt remove package-name

# Clean unused packages
sudo apt autoremove
    \end{lstlisting}

    \subsection*{RHEL-based (CentOS, Fedora)}
    \begin{lstlisting}[style=linuxStyle]
# Install a package
sudo yum install package-name

# Remove a package
sudo yum remove package-name
    \end{lstlisting}

    \subsection*{Arch-based (Manjaro, Arch)}
    \begin{lstlisting}[style=linuxStyle]
# Install a package
sudo pacman -S package-name

# Remove a package
sudo pacman -R package-name
    \end{lstlisting}

    \section*{SSH & Remote Access}
    \begin{lstlisting}[style=linuxStyle]
# Connect to a remote server
ssh user@remote_host

# Copy files to/from remote server
scp file.txt user@remote_host:/path/
scp user@remote_host:/path/file.txt .

# Transfer files with rsync
rsync -avz file.txt user@remote_host:/path/
    \end{lstlisting}

    \section*{Archiving & Compression}
    \begin{lstlisting}[style=linuxStyle]
# Create a tar archive
tar -cvf archive.tar mydir/

# Extract a tar archive
tar -xvf archive.tar

# Create a compressed archive
tar -czvf archive.tar.gz mydir/

# Extract a compressed archive
tar -xzvf archive.tar.gz

# Compress a file
gzip file.txt
bzip2 file.txt

# Decompress a file
gunzip file.txt.gz
bunzip2 file.txt.bz2
    \end{lstlisting}

    \section*{Shell Scripting Basics}
    \begin{lstlisting}[style=linuxStyle]
#!/bin/bash
echo "Hello, World!"

# Variables
name="Linux"
echo "Welcome to $name!"

# Loops
for i in {1..5}; do
  echo "Loop $i"
done

# Conditionals
if [ -f "file.txt" ]; then
  echo "File exists"
else
  echo "File does not exist"
fi

# Read user input
read -p "Enter your name: " user
echo "Hello, $user!"
    \end{lstlisting}

    \section*{Crontab (Scheduled Jobs)}
    \begin{lstlisting}[style=linuxStyle]
# List scheduled jobs
crontab -l

# Edit crontab
crontab -e

# Run a script every day at 2 AM
0 2 * * * /path/to/script.sh
    \end{lstlisting}

    \section*{System Information}
    \begin{lstlisting}[style=linuxStyle]
# Show Linux distribution
lsb_release -a

# Kernel version
uname -r

# List hardware info
lshw

# Show environment variables
env
    \end{lstlisting}

\end{document}
