% Preamble
\documentclass[a4paper, 11pt]{book}

% Packages
\usepackage{amsmath}    % For math symbols and equations
\usepackage{graphicx}   % For including graphics
\usepackage{geometry}   % For adjusting page layout
\usepackage{fancyhdr}   % For custom headers/footers
\usepackage{hyperref}   % For hyperlinks
\usepackage{listings}   % For code listing
\usepackage{lipsum}     % For placeholder text (to test layout)
\usepackage{helvet} % Helvetica as a sans-serif font alternative\usepackage{xcolor}
\usepackage{enumitem} %Allow for lists

\renewcommand{\rmdefault}{phv} % Set the default font family to Helvetica

% Set page margins (A4 paper)
\geometry{top=1in, bottom=1in, left=1in, right=1in}

% Set up the header
\pagestyle{fancy}
\fancyhf{}
\fancyhead[L]{Linux Quick Reference}
%\fancyhead[C]{Your Name}
\fancyhead[R]{\thepage}

% Document
\begin{document}
    % Title Page
    \begin{titlepage}
        \centering
        \vspace*{2in}
        \Huge \textbf{Linux Quick Reference}
        \vfill
        \Large -----
        \vfill
%        \Large Date:% \today
    \end{titlepage}

% Table of Contents
    \tableofcontents
    
    \section{Introduction}

    \paragraph{}
    The purpose of these notes/PDF is to provide me with a quick reference for useful commands when working with Linux machines.
    
    \section{Linux}

\end{document}