%! Author = Marjan
%! Date = 02/02/2025
% Preamble
\documentclass[a4paper, 11pt]{book}

% Packages
\usepackage{amsmath}    % For math symbols and equations
\usepackage{graphicx}   % For including graphics
\usepackage{geometry}   % For adjusting page layout
\usepackage{fancyhdr}   % For custom headers/footers
\usepackage{hyperref}   % For hyperlinks
\usepackage{listings}   % For code listing
\usepackage{lipsum}     % For placeholder text (to test layout)
\usepackage{helvet} % Helvetica as a sans-serif font alternative\usepackage{xcolor}
\usepackage{enumitem} %Allow for lists

\renewcommand{\rmdefault}{phv} % Set the default font family to Helvetica

% Set page margins (A4 paper)
\geometry{top=1in, bottom=1in, left=1in, right=1in}

% Set up the header
\pagestyle{fancy}
\fancyhf{}
\fancyhead[L]{Linux and Tooling notes}
\fancyhead[C]{Your Name}
\fancyhead[R]{\thepage}

% Document
\begin{document}
    % Title Page
    \begin{titlepage}
        \centering
        \vspace*{2in}
        \Huge \textbf{Linux and Tooling Notes}
        \vfill
        \Large -----
        \vfill
        \Large Date:% \today
    \end{titlepage}

% Table of Contents
    \tableofcontents
    \newpage

%\part{Linux}


    \chapter{Linux}


    \section{Useful programs}


    \section{Notes about the file structure}


    \section{Bash}
    Reminders on Bash related things

    \subsection{Bash CLI}
    Useful commands here

    \subsection{Bash Scripting}
    Hmmmmmm useful stuff for scripting


    \section{Hardening a Linux box}
    Hardening a linux box

%\part{The rest}


    \chapter{CI/CD stuff}


    \section{Git}

    Niche git commands


    \section{Docker}

    \paragraph{Motivations}
    Solves the problem of setting up environments for applications.

    \paragraph{Operating Systems, Hardware and the Kernel}
    Note: Some of these features are Linux specific......
    Note: Docker is a Linux virtual machine, inside this Linux VM is where the containers are running.
    This VM uses the Linux kernel, and it will be doing \textit{Linux} kernel things.

    Understanding what a Kernel is.
    Software process that acts as the middle guy between programs running on a computer and hardware resources.
    Programs interact with the kernel through system calls.
    Kernel exposes different system calls.
    Namespaces..segment hardware resources to different processes, control groups are another term, limit the CPU, RAM etc that a program can use.

    Namespacing (Linux specific) - Isolating resources per process (or group of processes) - Processes, Hard Drive, Network, Users, Hostnames, Inter Process Communication

    Control groups - Limit amount of processes used per process ---- Memory, CPU, HD I/O, Network Bandwith

    Container - a process or set or processes that have a grouping of resources specifically assigned to it

    Image - A file system snapshot of a very specific set of directories and files.
    Will also contain a specific startup command.

    Kernel will isolate a section of the hardrive/physical resource and make it avaiable only to this container.
    This container has access to this very specifc segment of the physical resources.


    \paragraph{Terminology}
    Docker is an overloaded term.
    Could be referring to Docker client, docker server, docker hub or docker compose.
    The goal of these tools is to create and run containers.

    \paragraph{Image}
    Single file with all the deps and config required to run a program.
    Containers are an instance of a program.

    \paragraph{Container}
    A program with its own isolated set of hardware resources. Consider kernel.

    \paragraph{Docker Client}
    Tool commands are issued to

    \paragraph{Docker Server}
    Tool that handles the docker related tasks such as creating images, running containers etc.
    Docker will check if a local image exists before downloading one from a central repository.


    \textbf{The Docker Ecosystem}

    \begin{itemize}[label=-]
        \item Docker Client - The CLI that interacts with the server
        \item Docker Server/Daemon - Background Server running on a machine
        \item Docker Machine - Used to manage docker environments
        \item Docker Images - The files used to create containers
        \item Docker Hub - A public repository for storing images
        \item Docker Compose - Useful for multi container environments, most applications will require different services.
    \end{itemize}


    \section{Docker commands}

    \subsubsection{Docker compose}


    \section{Jenkins}


    \section{Kubernetes}


    \section{Ansible}
    Ansible for now. Need to consider chef and puppet


    \section{Terraform}
    Cloud stuff here


    \chapter{JVM Specific / Delete and or separate}


    \section{Maven}
    Maven stuff or keep separated


    \section{JVM details}
    Maybe include some low level JVM stuff here....or keep separate


    \chapter{Hmm maybe more, maybe delete}
    Other tools that can be used via a shell

    \paragraph{Code Example}


    \section{Code Example}
    \begin{lstlisting}[language=Java, caption=Java Code for a Simple Cache]
public class SimpleCache {
    private Map<String, String> cache = new HashMap<>();

    public String get(String key) {
        return cache.get(key);
    }

    public void put(String key, String value) {
        cache.put(key, value);
    }
}
    \end{lstlisting}

\end{document}
